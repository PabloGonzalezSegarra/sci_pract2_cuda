% ========== CÓDIGO FUENTE (C/C++/CUDA) ==========
% Configuración de listings para mostrar snippets de código
%
% Uso:
%   Inline: \code{int x = 5;} o \cudacode{__global__ void f()}
%   Bloque:
%     \begin{lstlisting}[language=CUDA, caption={Mi kernel}]
%     __global__ void kernel() { ... }
%     \end{lstlisting}
%   Desde archivo:
%     \lstinputlisting[language=CUDA, caption={...}]{../archivo.cu}

\usepackage{listings}
\usepackage{xcolor}

% Colores para el código
\definecolor{codegreen}{rgb}{0,0.6,0}
\definecolor{codegray}{rgb}{0.5,0.5,0.5}
\definecolor{codepurple}{rgb}{0.58,0,0.82}
\definecolor{codeblue}{rgb}{0.13,0.13,1}
\definecolor{codeorange}{rgb}{0.8,0.4,0}
\definecolor{backcolour}{rgb}{0.97,0.97,0.97}

% Estilo base para código
\lstdefinestyle{codestyle}{
    backgroundcolor=\color{backcolour},
    commentstyle=\color{codegreen}\itshape,
    keywordstyle=\color{codeblue}\bfseries,
    numberstyle=\tiny\color{codegray},
    stringstyle=\color{codeorange},
    basicstyle=\ttfamily\footnotesize,
    breakatwhitespace=false,
    breaklines=true,
    captionpos=b,
    keepspaces=true,
    numbers=left,
    numbersep=8pt,
    showspaces=false,
    showstringspaces=false,
    showtabs=false,
    tabsize=4,
    frame=single,
    framesep=3pt,
    xleftmargin=15pt,
    framexleftmargin=15pt,
}

% Definir lenguaje CUDA (extiende C++)
\lstdefinelanguage{CUDA}[]{C++}{
    morekeywords={__global__, __device__, __host__, __shared__, __constant__,
                  __restrict__, __align__, __launch_bounds__,
                  dim3, cudaError_t, cudaEvent_t, cudaStream_t,
                  blockIdx, blockDim, threadIdx, gridDim, warpSize,
                  cudaMalloc, cudaFree, cudaMemcpy, cudaMemcpyAsync,
                  cudaMemcpyHostToDevice, cudaMemcpyDeviceToHost,
                  cudaMemcpyDeviceToDevice, cudaMemcpyHostToHost,
                  cudaDeviceSynchronize, cudaGetLastError, cudaGetErrorString,
                  cudaEventCreate, cudaEventDestroy, cudaEventRecord,
                  cudaEventSynchronize, cudaEventElapsedTime,
                  cudaGetDeviceCount, cudaGetDeviceProperties, cudaSetDevice,
                  atomicAdd, atomicSub, atomicExch, atomicMin, atomicMax,
                  __syncthreads, __threadfence, __syncwarp,
                  sinf, cosf, sqrtf, expf, logf, powf, fabsf},
    morecomment=[l]{//},
    morecomment=[s]{/*}{*/},
    morestring=[b]",
    sensitive=true,
}

% Configuración por defecto
\lstset{style=codestyle}

% Comandos rápidos para código inline y bloques
% Uso: \code{int x = 5;}
\newcommand{\cppcode}[1]{\lstinline[language=C++]|#1|}

% Uso: \cudacode{__global__ void kernel()}
\newcommand{\cudacode}[1]{\lstinline[language=CUDA]|#1|}
