\documentclass[a4paper,12pt]{article}
\usepackage[spanish]{babel}

% Packages - Plantilla
\usepackage{graphicx}
\usepackage{amsmath}
\usepackage{geometry}
\usepackage{fancyhdr}
\usepackage{setspace}
\usepackage{titlesec}  % For title formatting
\usepackage{tocloft}
\usepackage[colorlinks=true,linkcolor=black,citecolor=black,urlcolor=blue]{hyperref}
\usepackage{enumitem}
\setlist[itemize]{topsep=0pt, label=\textbullet, itemsep=0pt}
\usepackage{caption}
\captionsetup[table]{name=Tabla}

% Packages - User
% ========== CÓDIGO FUENTE (C/C++/CUDA) ==========
% Configuración de listings para mostrar snippets de código
%
% Uso:
%   Inline: \code{int x = 5;} o \cudacode{__global__ void f()}
%   Bloque:
%     \begin{lstlisting}[language=CUDA, caption={Mi kernel}]
%     __global__ void kernel() { ... }
%     \end{lstlisting}
%   Desde archivo:
%     \lstinputlisting[language=CUDA, caption={...}]{../archivo.cu}

\usepackage{listings}
\usepackage{xcolor}

% Colores para el código
\definecolor{codegreen}{rgb}{0,0.6,0}
\definecolor{codegray}{rgb}{0.5,0.5,0.5}
\definecolor{codepurple}{rgb}{0.58,0,0.82}
\definecolor{codeblue}{rgb}{0.13,0.13,1}
\definecolor{codeorange}{rgb}{0.8,0.4,0}
\definecolor{backcolour}{rgb}{0.97,0.97,0.97}

% Estilo base para código
\lstdefinestyle{codestyle}{
    backgroundcolor=\color{backcolour},
    commentstyle=\color{codegreen}\itshape,
    keywordstyle=\color{codeblue}\bfseries,
    numberstyle=\tiny\color{codegray},
    stringstyle=\color{codeorange},
    basicstyle=\ttfamily\footnotesize,
    breakatwhitespace=false,
    breaklines=true,
    captionpos=b,
    keepspaces=true,
    numbers=left,
    numbersep=8pt,
    showspaces=false,
    showstringspaces=false,
    showtabs=false,
    tabsize=4,
    frame=single,
    framesep=3pt,
    xleftmargin=15pt,
    framexleftmargin=15pt,
}

% Definir lenguaje CUDA (extiende C++)
\lstdefinelanguage{CUDA}[]{C++}{
    morekeywords={__global__, __device__, __host__, __shared__, __constant__,
                  __restrict__, __align__, __launch_bounds__,
                  dim3, cudaError_t, cudaEvent_t, cudaStream_t,
                  blockIdx, blockDim, threadIdx, gridDim, warpSize,
                  cudaMalloc, cudaFree, cudaMemcpy, cudaMemcpyAsync,
                  cudaMemcpyHostToDevice, cudaMemcpyDeviceToHost,
                  cudaMemcpyDeviceToDevice, cudaMemcpyHostToHost,
                  cudaDeviceSynchronize, cudaGetLastError, cudaGetErrorString,
                  cudaEventCreate, cudaEventDestroy, cudaEventRecord,
                  cudaEventSynchronize, cudaEventElapsedTime,
                  cudaGetDeviceCount, cudaGetDeviceProperties, cudaSetDevice,
                  atomicAdd, atomicSub, atomicExch, atomicMin, atomicMax,
                  __syncthreads, __threadfence, __syncwarp,
                  sinf, cosf, sqrtf, expf, logf, powf, fabsf},
    morecomment=[l]{//},
    morecomment=[s]{/*}{*/},
    morestring=[b]",
    sensitive=true,
}

% Configuración por defecto
\lstset{style=codestyle}

% Comandos rápidos para código inline y bloques
% Uso: \code{int x = 5;}
\newcommand{\cppcode}[1]{\lstinline[language=C++]|#1|}

% Uso: \cudacode{__global__ void kernel()}
\newcommand{\cudacode}[1]{\lstinline[language=CUDA]|#1|}
  % Configuración para snippets de código C/C++/CUDA

% Información de la práctica
\newcommand{\autor}{Pablo González Segarra}
\newcommand{\titulo}{Práctica 2}
\newcommand{\subtitulo}{Parte 1: Compilando Programas en C}
\newcommand{\fecha}{\today}
\newcommand{\titulacion}{Máster Universitario en Ingeniería de Computadores y Redes}

% Header and Footer
\pagestyle{fancy}
\fancyhf{}
\fancyhead[L]{\titulo}
\fancyhead[R]{\thepage}
\setlength{\headheight}{15pt}

% Other configurations
\geometry{margin=1in}
\setlength{\parskip}{0.5\baselineskip}
\setlength{\cftbeforesecskip}{0pt} % Espacio entre entradas del índice

% Tittle page
% Title Formatting
\titleformat{\section}{\normalfont\Large\bfseries}{\thesection}{1em}{}

\author{\autor}
\date{}

% Cover Page
\title{
    \pagestyle{empty}
    
    \vspace{-2cm} % Adjust vertical space

    \begin{figure}[ht]
        \centering
        \begin{minipage}[t]{0.3\textwidth}
            \includegraphics[width=\linewidth]{tittle/UPV-Logo.png}
            % \caption{Descripción de la imagen 1}
        \end{minipage}%
        \hfill
        \begin{minipage}[t]{0.48\textwidth}
            \vspace{-2cm}
            \includegraphics[width=\linewidth]{tittle/ETSINF-Logo.png}
            
            % \caption{Descripción de la imagen 2}
        \end{minipage}
    \end{figure}

    \vspace{0.3cm} % Adjust vertical space after the logo
    
    \textbf{
    \Huge \titulo  \\% Título
    {\Large \subtitulo}}\\ % Subtítulo
    
    \vspace{0.6cm} % Adjust vertical space
    \large \titulacion \\
    \vspace{0.3cm} % Adjust vertical space
    \large \fecha
}


% Document
\begin{document}
% Title Page
\maketitle
\thispagestyle{empty}
\vfill

% Table of Contents
\setcounter{page}{1}  % Start counting from 1
\tableofcontents
\newpage

% 1
\section*{Ejercicio 1 y 2}\addcontentsline{toc}{section}{Ejercicio 1}
Debido a la similitud entre los ejercicios 1 y 2, se presentan juntos en esta sección.
A continuación se muestra el código modificado necesario para ejecutar el programa
en la GPU utilizando CUDA y comprobar que los resultados son correctos.
\lstinputlisting[
    language=CUDA, 
    style=codestyle, 
    caption={Código CUDA modificado para ejecutar en GPU}
    ]
    {../ej_1_y_2.cu}

% 3
\section*{Ejercicio 3}\addcontentsline{toc}{section}{Ejercicio 3}

El codigo modificado para poder obtener los tiempos de ejecución es el siguiente:

\lstinputlisting[
    language=CUDA, 
    style=codestyle, 
    caption={Código CUDA modificado para medir tiempos de ejecución}
    ]
    {../ej3.cu}

Se ha ejecutado 5 veces para evitar variaciones ajenas al rendimiento del código. El tiempo de 
ejecucion tiene en cuenta tanto la transferencia de datos entre host y device como la 
ejecución del kernel.

Resultado de las ejecuciones:
\begin{enumerate}[topsep=0pt,itemsep=0pt]
    \item 5887.4 ms
    \item 5864.1 ms
    \item 5918.6 ms
    \item 5856.9 ms
    \item 5868.1 ms
\end{enumerate}

Siendo el mejor tiempo 5856.9 ms.

% 4
\section*{Ejercicio 4}\addcontentsline{toc}{section}{Ejercicio 4}

Para comparar la aceleracion obtenida al ejecutar el código en la GPU frente a la CPU,
primero se ha compilado y ejecutado la version original en CPU, obteniendo un tiempo mejor de
992.6 ms.

Para obtener los tiempos de ejecución en GPU con distintas configuraciones numero de bloques
y hilos por bloque, se ha modificado el código del ejercicio 3 para variar estos parámetros. Todas las 
configuraciones se han ejecutardo

\begin{table}[h!]
\centering
\begin{tabular}{lcc}
    \hline
	extbf{Versión} & \textbf{Tiempo de ejecución (ms)} & \textbf{Speedup GPU vs CPU} \\
    \hline
    1 thread / block & & \\
    16 threads / block & & \\
    32 threads / block & & \\
    256 threads / block & & \\
    \hline
\end{tabular}
\caption{Tiempos de ejecución y speedup para distintas configuraciones de hilos/ bloque}
\label{tab:tiempos-speedup}
\end{table}
 
% 5
\section*{Ejercicio 5}\addcontentsline{toc}{section}{Ejercicio 5}

% 6
\section*{Ejercicio 6}\addcontentsline{toc}{section}{Ejercicio 6}

% 7
\section*{Ejercicio 7}\addcontentsline{toc}{section}{Ejercicio 7}



\cleardoublepage
\pagestyle{plain}
% Bibliography (if required)
% \bibliographystyle{ieeetr}
% \bibliography{references.bib}  % Add a .bib file if you have references

\end{document}
